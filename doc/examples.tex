% \linenumbers


\clearpage%\mbox{}\clearpage

\section{Introduction}\label{intro}
% \lipsum[1]

(1) The original technique to execute an arbitrary command:
\begin{verbatim}
\immediate\write18{node
  "\CXLTXmainroute"
  "\currfilepath"
  "helo"
  "readers (one)"
  > /tmp/temp.dat}\input{/tmp/temp.dat}
\end{verbatim}

(2) With ugly details largely hidden, the \verb#\exec{}# command is still fully general:
\begin{verbatim}
\exec{node
  "\CXLTXmainroute"
  "\currfilepath"
  "helo"
  "readers (two)"}
\end{verbatim}

(3) \verb#\noderunscript{}# will execute NodeJS code that adheres to the call convention established by
\CXLTX:
\begin{verbatim}
\noderunscript
  {\CXLTXmainroute}
  {\currfilepath}
  {helo}
  {readers (three)}
\end{verbatim}

(4) Like the previous example, but with standard values assumed as shown above. This is the form that you
will want to use most of the time:
\begin{verbatim}
\noderun{helo}{readers (four)}
\end{verbatim}


{\textbf{Outputs:}}

\immediate\write18{node "\CXLTXmainroute" "\currfilepath" "helo" "readers (one)" > /tmp/temp.dat}\input{/tmp/temp.dat}

\exec{node "\CXLTXmainroute" "\currfilepath" "helo" "readers (two)"}

\noderunscript{\CXLTXmainroute}{\currfilepath}{helo}{readers (three)}

\noderun{helo}{readers (four)}

% ==========================================================================================================
\clearpage\section{Configuration}\label{config}

To use \CXLTX, put
\begin{verbatim}
\usepackage{coffeexelatex}
\end{verbatim}
into the header section of your \LaTeX\ file.

You may also want to include these lines in your \LaTeX\ document; these define, respectively, the route
to the \CXLTX\ executable (\verb#coffeexelatex/lib/main.js#) and the temporary file that is used to communicate
between \TeX\ and your scripts (relative routes are resolved with respect to the current working directory,
so if you set a relative route, you must always run \TeX\ from within the same directory):
\begin{verbatim}
\renewcommand{\CXLTXmainroute}{../../lib/main}
\renewcommand{\CXLTXtempoutroute}{/tmp/coffeexelatex.tex}
\end{verbatim}


% % ==========================================================================================================
% \clearpage\section{Geometry}\label{geo}

% % Write arbitrary text into the aux file:
% \auxc{this line goes to the aux file}

% % Write geometry data into the aux file:
% \auxgeo

% Use geometry data from aux file to render a table of layout dimensions into the document;
% note the we could have used the \verb#\auxgeo# command anywhere in the document and that
% this currently only works for documents with a single, constant layout.

% Also note we're using a dash instead of an underscore here—in \TeX, underscores are special, so
% we conveniently allow dashes to make things easier. The \CXLTX\ command \verb#show-geometry# does
% not take arguments, which is why the second pair of braces has been left empty:

% {\textbf{Command:}}
% \begin{verbatim}
% \noderun{show-geometry}{}
% \end{verbatim}

% {\textbf{Output:}}

% \noderun{show-geometry}{}

% ==========================================================================================================
\clearpage\section{Character Escaping}\label{esc}

The \CXLTX\ command \verb#show-special-chrs# demonstrates that it is easy to include \TeX\ special characters
in the return value. The simple rule is that whenever the output of a command is meant to be understood
literally, it should be \verb#@escape#d:

{\textbf{Command:}}

\begin{verbatim}
\noderun{show-special-chrs}{}
\end{verbatim}

{\textbf{Output:}}

\noderun{show-special-chrs}{}


% ==========================================================================================================
\clearpage\section{The \texttt{aux} Object}\label{aux}



% ----------------------------------------------------------------------------------------------------------
\subsection{Labels}\label{labels}

\CXLTX\ will try and collect all labels from the \verb#*.aux# file associated with the current job; from
inside your scripts 〓 〓 〓 〓 〓 〓 〓 〓 〓 〓 〓 〓 〓 〓 〓 〓 〓 〓 〓

{\textbf{Command:}}

\begin{verbatim}
\noderun{show-aux}{}
\end{verbatim}

{\textbf{Output:}}

{\fontsize{3mm}{3mm}\noderun{show-aux}{}}


% ----------------------------------------------------------------------------------------------------------
\subsection{Evaluating Expressions}\label{evalcs}

The commands \verb#\evalcs{}# and \verb#\evaljs{}# allow you to evaluate an arbitrary self-contained
expression, written either in CoffeeScript or in JavaScript:

{\textbf{Command:}}

\begin{verbatim}
$23 + 65 * 123 = \evalcs{23 + 65 * 123}$
\end{verbatim}

{\textbf{Output:}}

$23 + 65 * 123 = \evalcs{23 + 65 * 123}$


% ==========================================================================================================
\clearpage\section{Curl}\label{curl}
\begin{verbatim}
\curlRaw{127.0.0.1:8910/foobar.tex/helo/friends}
\end{verbatim}
\curlRaw{127.0.0.1:8910/foobar.tex/helo/friends}



% ----------------------------------------------------------------------------------------------------------
\subsection{Unicode}\label{unicode}

\begin{verbatim}
\curl{helo}{黎永強}
\end{verbatim}

\curl{helo}{黎永強}

\begin{verbatim}
\curl{helo}{äöüÄÖÜß}
\end{verbatim}

\curl{helo}{äöüÄÖÜß}

% ----------------------------------------------------------------------------------------------------------
\subsection{The URL environment}\label{urlenv}

Typing URLs in \LaTeX\ can be quite a chore, given the number of active and otherwise `special' characters
to take care of: not only does \TeX\ itself define some special characters, not only do the RFCs that govern
URL syntax consider 〓 〓 〓 〓 〓 〓 〓 〓 〓 〓 〓 〓 〓 〓 〓 〓 〓 〓 〓 special—when we communicate with our \CXLTX\ server,
we do so by executing a \verb#curl ...# command via the OS shell (normally \verb#sh#), which again has its own
rich set of specials. In order to alleviate the burden on the casual user, we define a new environment,
`URL`, that somewhat simplifies writing (parts of) URLs:
\begin{verbatim}
\begin{URL}
\curl{helo}{`\{ [ $ ~ % \# ^ | \} ] '?}
\end{URL}
\end{verbatim}
\begin{URL}
\curl{helo}{`\{ [ $ ~ % \# ^ | \} ] '?}
\end{URL}






% \begin{verbatim}
% \begin{URL}
% \curl{helo}{B \& C % Dollar: $ hash: \# caret: ^ wave:~ backtick: \` }
% \end{URL}
% \end{verbatim}
% \begin{URL}
% \curl{helo}{B \& C % Dollar: $ hash: \# caret: ^ wave:~ backtick: \` }
% \end{URL}

% \usepackage{textcomp} \textquotesingle \textasciigrave


% \CXLTXtemperrroute\ is \CXLTXiffileempty{\CXLTXtemperrroute}{indeed}{not} empty.
% \CXLTXtempoutroute\ is \CXLTXiffileempty{\CXLTXtempoutroute}{indeed}{not} empty.

