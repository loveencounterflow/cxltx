% \linenumbers


\section{Examples}\label{examples}
% \lipsum[1]

\subsection{Spawning NodeJS}\label{spawningnodejs}

(1) The original technique to execute an arbitrary command:
\begin{verbatim}
\immediate\write18{node
  "\CXLTXmainRoute"
  "\currfileabsdir"
  "\jobname"
  "helo"
  "readers (one)"
  > /tmp/temp.dat}\input{/tmp/temp.dat}
\end{verbatim}

(2) With ugly details largely hidden, the \verb#\exec{}# command is still fully general:
\begin{verbatim}
\exec{node
  "\CXLTXmainRoute"
  "\currfileabsdir"
  "\jobname"
  "helo"
  "readers (two)"}
\end{verbatim}

(3) \verb#\nodeRunScript{}# will execute NodeJS code that adheres to the call convention established by
\CXLTX:
\begin{verbatim}
\nodeRunScript
  {\CXLTXmainRoute}
  {\currfileabsdir}
  {\jobname}
  {helo}
  {readers (three)}
\end{verbatim}

(4) Like the previous example, but with standard values assumed as shown above. This is the form that you
will want to use most of the time (if you want to use the CL/RCI at all):
\begin{verbatim}
\nodeRun{helo}{readers (four)}
\end{verbatim}


{\textbf{Outputs:}}

\immediate\write18{node "\CXLTXmainRoute" "\currfileabsdir" "\jobname" "helo" "readers (one)" > /tmp/temp.dat}\input{/tmp/temp.dat}

\exec{node "\CXLTXmainRoute" "\currfileabsdir" "\jobname" "helo" "readers (two)"}

\nodeRunScript{\CXLTXmainRoute}{\currfileabsdir}{\jobname}{helo}{readers (three)}

\nodeRun{helo}{readers (four)}


% ----------------------------------------------------------------------------------------------------------
\subsection{Evaluating Expressions}\label{evalcs}

The commands \verb#\evalcs{}# and \verb#\evaljs{}# allow you to evaluate an arbitrary self-contained
expression, written either in CoffeeScript or in JavaScript:

\begin{verbatim}
$23 + 65 * 123 = \evalcs{23 + 65 * 123}$
\end{verbatim}

$23 + 65 * 123 = \evalcs{23 + 65 * 123}$


% ==========================================================================================================
\subsection{Spawning cURL}\label{spawningcurl}
\begin{verbatim}
\curlRaw{127.0.0.1:8910/foobar.tex/\jobname/helo/friends}
\end{verbatim}
\curlRaw{127.0.0.1:8910/foobar.tex/\jobname/helo/friends}


% ==========================================================================================================
\subsection{Character Escaping}\label{esc}

The \CXLTX\ command \verb#show-special-chrs# demonstrates that it is easy to include \TeX\ special characters
in the return value. The simple rule is that whenever the output of a command is meant to be understood
literally, it should be \verb#@escape#d:

\begin{verbatim}
\nodeRun{show-special-chrs}{}
\end{verbatim}

\nodeRun{show-special-chrs}{}


% ----------------------------------------------------------------------------------------------------------
\subsection{Unicode}\label{unicode}

The next few examples demonstrate that Unicode characters---even ones from outside the Unicode Basic
Multilingual plain, which frequently cause difficulties---can be transported to and from the server
without losses or Mojibake / squiggles:

\begin{verbatim}
\curl{helo}{äöüÄÖÜß}
\end{verbatim}

\curl{helo}{äöüÄÖÜß}

Chinese characters from the Unicode BMP (`16 bit'):

\begin{verbatim}
\curl{helo}{黎永強}
\end{verbatim}

\curl{helo}{黎永強}

Chinese characters from the Unicode SIP (`32 bit'---these needed a little trick to make \XeLaTeX\ choose
the right font; see \verb#coffeexelatex.sty#):

\begin{verbatim}
\curl{helo}{𠀀𠀐𠀙}
\end{verbatim}

\curl{helo}{𠀀𠀐𠀙}

% % ----------------------------------------------------------------------------------------------------------
% \subsection{The URL environment}\label{urlenv}

% Typing URLs in \LaTeX\ can be quite a chore, given the number of active and otherwise `special' characters
% to take care of: not only does \TeX\ itself define some special characters, not only do the RFCs that govern
% URL syntax consider 〓 〓 〓 〓 〓 〓 〓 〓 〓 〓 〓 〓 〓 〓 〓 〓 〓 〓 〓 special---when we communicate with our \CXLTX\ server,
% we do so by executing a \verb#curl ...# command via the OS shell (normally \verb#sh#), which again has its own
% rich set of specials. In order to alleviate the burden on the casual user, we define a new environment,
% `URL`, that somewhat simplifies writing (parts of) URLs:
% \begin{verbatim}
% \begin{URL}
% \curl{helo}{`\{ [ $ ~ % \# ^ | \} ] '?}
% \end{URL}
% \end{verbatim}
% \begin{URL}
% \curl{helo}{`\{ [ $ ~ % \# ^ | \} ] '?}
% \end{URL}



% \evalcs{ ( n * n for n in [ 1, 2, 3 ] ) }
% % [1,4,9]

% % \evalcs{ '\$' + ( 3 + 4 ) + '\$' }

% \currfileabsdir

% \currfileabspath

% {\fontsize{3mm}{3mm}\nodeRun{show-aux}{}}


% % \begin{verbatim}
% % \begin{URL}
% % \curl{helo}{B \& C % Dollar: $ hash: \# caret: ^ wave:~ backtick: \` }
% % \end{URL}
% % \end{verbatim}
% % \begin{URL}
% % \curl{helo}{B \& C % Dollar: $ hash: \# caret: ^ wave:~ backtick: \` }
% % \end{URL}

% % \usepackage{textcomp} \textquotesingle \textasciigrave


% ==========================================================================================================
\subsection{The \texttt{aux} Object}\label{aux}

To facilitate data exchange between the \TeX\ process and the server, \CXLTX\ provides facilities
to read and write data from and to the \verb#aux# file assoated with the current job:

\begin{itemize}
\item
  the \TeX\ commands \verb#\aux#, \verb#\auxc#, \verb#\auxcs#, and \verb#\auxpod#, which write to the
  \verb#aux# file;
\item
  the method \verb#CXLTX.main.read_aux = ( handler ) ->#, which reads (and parses) data written with one of
  the above commands.
\end{itemize}

% ----------------------------------------------------------------------------------------------------------
\subsubsection{The \texttt{\textbackslash{}aux*} Commands}\label{aux}

Because they're quite straightforward, let's have a look at the actual definitions of the
\verb#aux*# commands:

\begin{verbatim}
\makeatletter
\catcode`\%=11
\newcommand{\aux}[1]{\immediate\write\@auxout{#1}}
\newcommand{\auxc}[1]{\immediate\write\@auxout{% #1}}
\newcommand{\auxcs}[1]{\immediate\write\@auxout{% coffee #1}}
\newcommand{\auxpod}[2]{\immediate\write\@auxout{% coffee #1: \{ #2 \}}}
\catcode`\%=14
\makeatother
\end{verbatim}

We see our old friend \verb#\immediate\write# here, this time accessing channel \verb#\@auxout#. All commands
will write a single line to the \verb#aux# file.

\begin{itemize}
\item
  \verb#\aux# is the most basic command and will write text as-is to the \verb#aux# file;
\item
  \verb#\auxc# puts whatever is written behind a \verb#%# (percent sign), so it appears as a comment when
  \TeX\ re-reads the \verb#aux# file;
\item
  \verb#\auxcs# writes text behind a \verb#% coffee# marker, facilitating recognition on the server side;
\item
  \verb#\auxpod# takes a name and a CoffeeScript Plain Old Dictionary literal ({\em without} the braces)
  to the \verb#aux# file; to the server, this will become available as \verb#@aux[ name ]#;
\end{itemize}

The \verb#\auxgeo# / \verb#@\curl{show-geometry}{}# command pair is a good example how to use \verb#\auxpod#.


% ----------------------------------------------------------------------------------------------------------
\subsubsection{Geometry}\label{geo}

% Write arbitrary text into the aux file:
\auxc{this line goes to the aux file}

% Write geometry data into the aux file:
\auxgeo

Use geometry data from aux file to render a table of layout dimensions into the document;
note the we could have used the \verb#\auxgeo# command anywhere in the document and that
this currently only works for documents with a single, constant layout.

Also note we're using a dash instead of an underscore here---in \TeX, underscores are special, so
we conveniently allow dashes to make things easier. The \CXLTX\ command \verb#show-geometry# does
not take arguments, which is why the second pair of braces has been left empty:

\begin{verbatim}
\curl{show-geometry}{}
\end{verbatim}

\curl{show-geometry}{}

After \verb#show-geometry# has been performed, the \verb#@aux# object has been populated with data from the
\verb#aux# file; it then looks like this for the current document:


\begin{verbatim}
\curl{show-aux}{}
\end{verbatim}

\curl{show-aux}{}


% ----------------------------------------------------------------------------------------------------------
\subsubsection{Labels}\label{labels}

As it stands, \CXLTX\ will try and collect all pertinent data from the \verb#@aux# file when
\verb#@read_aux# is called; this currently includes

labels from the \verb#*.aux# file associated with the current job; from
inside your scripts 〓 〓 〓 〓 〓 〓 〓 〓 〓 〓 〓 〓 〓 〓 〓 〓 〓 〓 〓

\begin{verbatim}
\curl{clear-aux}{}
\curl{show-aux}{}
\end{verbatim}

% {\fontsize{3mm}{3mm}\nodeRun{show-aux}{}}
% \nodeRun{show-aux}{}
\curl{clear-aux}{}
\curl{show-aux}{}
