\textbf{Att.} this part of the docs is only partially up to date; see
\texttt{doc/examples.tex} and \texttt{xltx-manual.pdf} for working code.

\section{C(offee)X(e)L(a)T(e)X}\label{coffeexelatex}

\subsection{What is it? And Why?}\label{what-is-it-and-why}

Everyone who has worked with LaTeX knows how hard it can often be to get
seemingly simple things done in this Turing-complete markup language.
Let's face it, (La)TeX has many problems; the
\href{http://www.infoq.com/presentations/Simple-Made-Easy}{complectedness}
of its inner workings and the extremely uneven syntax of its commands
put a heavy burden on the average user. The funny thing is that while
TeX is all about computational text processing, doing math and string
processing are \emph{really hard} to get right---or even get done at
all---in this environment (not to mention that TeX has no notion of
higher-order data types, such as, say, lists).

Often one wishes one could just do a simple calculation or build a
typesetting object from available data \emph{outside} of all that makes
LaTeX so difficult to get right. Turns out you can already do that, and
you don't have to recompile TeX.

Most of the time, running TeX means to author a source file and have the
TeX executable convert that into a PDF. Of course, this implies reading
and writing of files and executing binaries. Interestingly for us, both
capabilities---file access and command execution---are made available to
user-facing side of TeX: writing to a file happens via the
\texttt{\textbackslash{}write} command, while input from a file is done
with \texttt{\textbackslash{}input}; command execution repurposes
\texttt{\textbackslash{}write}, which may be called with the special
stream number \texttt{18} (internally, TeX does almost everything with
registers that are sometimes given symbolic names; it also enumerates
`channels' for file operations, and reserves \#18 for writing to the
command line and executing stuff). This is how the
\texttt{\textbackslash{}exec} command is defined in CoffeeXeLaTeX:

\begin{verbatim}
\newcommand{\coffeexelatextemproute}{/tmp/coffeexelatex.tex}

\newcommand{\exec}[1]{%
  \immediate\write18{#1 > \coffeexelatextemproute}
  \input{\coffeexelatextemproute}
  }
\end{verbatim}

This command says, in essence: given a single argument \texttt{\#1},
have the OS execute it as a command, and do that immediately (i.e.~not
at some arbitrary point in the future); redirect its output into the
temporary file \texttt{/tmp/coffeexelatex.tex}; then, read back the
contents of that file, and insert that text into the current TeX source.

With some TeXs, its possible to avoid the temporary file by using
\texttt{\textbackslash{}@@input\textbar{}"dir"}, but XeTeX as provided
by TeXLive 2013 does not allow that. As much as i'd like to eliminate
the somewhat unelegant temporary file (that brings one more possible
point of failure to the equation; remember to
\texttt{\textbackslash{}renewcommand\{\textbackslash{}coffeexelatextemproute\}\{some/temp/location.tex\}}
where deemed necessary), it also has one advantage: in case TeX should
halt execution becuase of an error, and that error is due to a script
with faulty output, you can conveniently review the problematic source
by opening the temporary file in your text editor.

\subsection{Security Considerations}\label{security-considerations}

Be aware that executing arbitrary code with a command line like
\texttt{xelatex -{}-enable-write18 -{}-shell-escape somefile.tex} is
inherently unsafe: a TeX file downloaded from somewhere could erase your
disk, access your email, or install a program on your computer. This is
what the \texttt{-{}-enable-write18} switch is for: it is by default
fully or partially disabled so an arbitrary TeX source gets limited
access to your computer.

If you're scared now, please hang on a second. I just want to tell you
\emph{you should be really, really scared}. Why? Because if you ever
downloaded some TeX source to compile it on your machine \textbf{even
without the \texttt{-{}-enable-write18} switch} you've already
\textbf{executed potentially harmful code}. Few people are aware of it,
but many TeX installations are quite `liberal' in respect to what TeX
sources---TeX programs, really---are allowed to do \emph{even in absence
of command line switches}, and, as a result, even people who are hosting
public TeX installations for a living are susceptible to malicious
code.*

\textbf{It is a misconception that TeX source is `safe' because `TeX is
text-based format'} (how stupid is that, anyway?); \textbf{the truth is
that by doing \texttt{latex xy.tex} you're executing code which may do
malicious things}. Period. That said, the papers linked below make it
quite clear that \texttt{-{}-enable-write18} just `opens the barn door',
as it were, but in fact, there are quite a few other and less well known
avenues for TeX-based malware to do things on your computer.

And please don't think you're safe just because you're not executing
anything but your own TeX source---that, in case you're using LaTeX, is
highly improbable: any given real-world LaTeX document will start with a
fair number of \texttt{\textbackslash{}usepackage\{\}} statements, and
each one of those refers to a source that is publicly accessible on the
internet and has been so for maybe five or ten or more years. Someone
might even have managed to place a mildly useful package on CTAN, one
that has some obfuscated parts designed to take over world leadership on
Friday, 13th---who knows? \textbf{The fact that TeX is a programming
language that works by repeatedly re-writing itself does not exactly
help in doing static code analysis}; in fact, such code is called
`\href{http://en.wikipedia.org/wiki/Metamorphic_code}{metamorphic code}'
and is a well-known technique employed by computer viruses.

I do not write this section of the present README to scare you away,
just to inform whoever is concerned of a little known fact of life. The
gist of this is: don't have \texttt{-{}-enable-write18} turned on except
you know what you're doing, but be aware that running TeX has always
been unsafe anyway.

\begin{quote}
*) see
e.g.~http://cseweb.ucsd.edu/\textasciitilde{}hovav/dist/tex-login.pdf,
http://cseweb.ucsd.edu/\textasciitilde{}hovav/dist/texhack.pdf
\end{quote}

\subsection{Installation}\label{installation}

\begin{itemize}
\itemsep1pt\parskip0pt\parsep0pt
\item
  put a symlink to your CoffeeXeLaTeX directory into a directory that is
  on LaTeX's search path; on OSX with LiveTeX, that could be
  \texttt{\textasciitilde{}/Library/texmf/tex/latex}.*
\end{itemize}

\begin{quote}
*: obviously, you could put your CoffeeXeLaTeX installation directly
there, but that strikes me as `wrong'.
\end{quote}

\subsection{Usage}\label{usage}

For a quick test, do

\begin{verbatim}
cd examples/example-1
perltex --nosafe --latex=xelatex example-1.tex
\end{verbatim}

from the command line; this should produce
\texttt{examples/example-1/example-1.pdf} (along with some other files).

You may want to have a look at
\texttt{examples/example-1/example-1.lgpl} to get an idea what exactly
happened behind the scenes (see
\href{https://www.tug.org/TUGboat/tb25-2/tb81pakin.pdf}{PerlTeX:
Defining LaTeX macros using Perl} for an overview of the process).

\subsection{Future Development}\label{future-development}

If CoffeeXeLaTeX turns out to be a useful tool, i can presently see the
following routes for development:

\begin{itemize}
\item
  Using a Perl shell-escape command to start \texttt{node} all over for
  each single JS/CS macro is doubly wasteful---first, a \texttt{perl}
  process is started which in turn starts a \texttt{node} subprocess.
  Depending on specific use, this can mean that two processes (none of
  them exactly lightweight) are initiated hundreds or thousands of times
  for a single document. This may be fixed in a number of ways:
\item
  Firstly, we could try and remove the Perl dependency, and call
  \texttt{node} in exactly the way that \texttt{perl} is called now. Not
  sure how to do that at this point in time.
\item
  Secondly, we could opt for a client / server model and make it so that
  instead of starting a (heavy) process for each JS/CS macro call, a
  (HTTP?) connection to a long-running NodeJS server is established.
  This is also attractive as it would simplify state keeping---as it
  stands, each call to a given macro starts with a clean slate (though
  one could imagine storing results from past calls in a file or a
  database).
\end{itemize}

Both of the above options are only worth implementing when it has been
shown that substantial benefits in terms of performance, easy of use,
and capabilities can be gained---something that is only meaningful after
experience with real-world use cases has been gained.

\begin{itemize}
\itemsep1pt\parskip0pt\parsep0pt
\item
  The one big incentive for using a 3rd-party language in tandem with
  LaTeX is to make things easy (or at least achievable) that are
  difficult (or impossible) using only LaTeX.
\end{itemize}

Regrettably, our efforts are still limited to what can be communicated
`over the wire' between the LaTeX process and the macro process; we do
not have direct access to (La)TeX internals as such, but must package
every pertinent facet of the ongoing typesetting process as a textual
argument for a macro.

Imagine you had to interact with your HTML page in this way---imagine
JavaScript in the browser was stateless and blissfully unaware of the
DOM and CSS, imagine HTML was Turing-complete-but-hard-to-use as is the
case with TeX. Your capabilities-improved web application page would be
littered with ugly
\texttt{\textless{}if condition='...'\textgreater{}...\textless{}/if\textgreater{}}
tags and circumlocutorily reified calls to JS. This is the state of
affairs of PerlTeX / CoffeeXeLaTeX, and it is definitely a programming
model begging to be improved.

\subsection{Related Work}\label{related-work}

\begin{itemize}
\item
  \href{http://www.pytex.org/}{PyTeX} (also dubbed QATeX) is a laudable
  effort that has, sadly, been stalling for around 11 years as of this
  writing (January 2014), so about everything of it is outdated. Their
  approach is apparently the opposite of what we do in CXLTX: they run
  TeX in daemon mode from Python; we have NodeJS start a server that
  listens to our TeX. Just for giggles, a quote from the above page:
  ``XML is hard work to key by hand. \emph{It lacks the mark-up
  minimization that SGML has}'' (my emphasis). Well, eleven years is a
  long time.
\item
  \href{https://github.com/gpoore/pythontex}{PythonTeX} is an
  interesting approach to bringing LaTeX and Python together.
  Unfortunately, the authors are preconcerned with showing off Pygment's
  syntax hiliting capabilities (which are \ldots{} not really that
  great) and how to print out integrals using SymPy, and fail to provide
  sample code of interest to a wider audience. Their copious 128-page
  manual only dwells for one and a half page on the topic of `how do i
  use this stuff', and that only to show off more SymPy capabilities.
  None of their sample code \emph{needs} PythonTeX anyway, since none of
  it demonstrates how to interact with the document typesetting process;
  as such, all their formulas and plots may be produced offline,
  independently from LaTeX. Given that the installation instructions are
  too scary and longwinded for my taste, and that PythonTeX is not part
  of LiveTeX, i've given up on the matter.
\end{itemize}

(the below taken from
http://get-software.net/macros/latex/contrib/pythontex):

\begin{itemize}
\item
  \href{http://www.ctan.org/tex-archive/macros/latex/contrib/sagetex}{SageTeX}
  allows code for the Sage mathematics software to be executed from
  within a \LaTeX~document.
\item
  Martin R. Ehmsen's
  \href{http://www.ctan.org/pkg/python}{\texttt{python.sty}} provides a
  very basic method of executing Python code from within a LaTeX
  document.
\item
  \href{http://elec.otago.ac.nz/w/index.php/SympyTeX}{SympyTeX} allows
  more sophisticated Python execution, and is largely based on a subset
  of SageTeX.
\item
  \href{http://www.luatex.org/}{LuaTeX} extends the pdfTeX engine to
  provide Lua as an embedded scripting language, and as a result yields
  tight, low-level Lua integration.
\end{itemize}

LuaTeX is one of the most interesting projects in this field as it
represents an attempt to provide a close coupling of a real programming
language with LaTeX. Unfortunately, that language is Lua, a language
that (like Go btw) believes that Unicode strings should be stored as
UTF-8 bytes. Equally unfortunately, LuaTeX uses pdfTeX, which can't
compare to XeLaTeX when it comes to using custom TTF/OTF fonts.

\subsection{Sample Command Lines}\label{sample-command-lines}

To make it so you can put a simple
\texttt{\textbackslash{}usepackage\{cxltx\}} inside your \texttt{tex}
files, do (on OSX)

\begin{verbatim}
cd ~/Library/texmf/tex/latex
ln -s route/to/cxltx cxltx
\end{verbatim}

This link `publishes' CXLTX to \texttt{kpathsea}, TeX's file search
tool.

Here is what i do to build \texttt{cxltx/cxltx-manual.pdf}:

\textbf{(1)} use \href{http://http://johnmacfarlane.net/pandoc}{Pandoc}
to convert \texttt{README.md} to \texttt{README.tex}:

\begin{verbatim}
pandoc -o cxltx/doc/README.tex cxltx/README.md
\end{verbatim}

\textbf{(2)} copy the \texttt{aux} file from the previous TeX
compilation step to preserve its data for CXLTX to see:

\begin{verbatim}
cp cxltx/doc/cxltx-manual.aux cxltx/doc/cxltx-manual.auxcopy
\end{verbatim}

\textbf{(3)} compile \texttt{cxltx-manual.tex} to
\texttt{cxltx-manual.pdf}:

\begin{verbatim}
# --enable-write18  allows to access external programs form within TeX#
# --halt-on-error   is a convenience so i don't have to type x on each TeX error
# --recorder        needed by the `currfile` package to get absolute routes

xelatex --output-directory cxltx/doc --halt-on-error --enable-write18 --recorder \
  cxltx/doc/cxltx-manual.tex
\end{verbatim}

\textbf{(4)} move the pdf file to its target location:

\begin{verbatim}
mv cxltx/doc/cxltx-manual.pdf cxltx
\end{verbatim}

\subsection{Useful Links}\label{useful-links}

http://www.ctan.org/tex-archive/macros/latex/contrib/perltex

http://ctan.space-pro.be/tex-archive/macros/latex/contrib/perltex/perltex.pdf

http://www.tug.org/TUGboat/tb28-3/tb90mertz.pdf

https://www.tug.org/TUGboat/tb25-2/tb81pakin.pdf
