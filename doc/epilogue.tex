
% ----------------------------------------------------------------------------------------------------------
\section{Epilogue}\label{epilogue}

\begingroup
\catcode`∆\active
\def∆{[\ldots]}
% \dotuline{dotty}
\em

We all know that {\em all's well that ends well}, but the astute reader will have noticed that through the lines
of this manual there is a certain amount of despair shining through.

There is a somehwat icky metaphor popular among \TeX nicians which compares the inner workings of \TeX\ to
the digestive work done by the mouth and the bowels, a metaphor that never fails to leave me both
unimpressed and uninformed. Kind souls answering the question of neophytes and noobies often draw on this
picture to explicate why some \verb#\foo{\bar}# invocation miserably flopped where the seemingly equally
innocuous \verb#\foo{\baz}# succeeded, causing \TeX\ to spit out unhelpfully obfuscated lines, often
ones spewn with @-signs, all decorated with a mysterious-slash-nonsensical error message and a line number
that quite frequently does {\em not} refer to the file currently being processed, before offering the user the
choice to either give up or else give up.\footnote{Reminds one of the now-classic Last Words Dialog featured in early versions of Microsoft Word (which in essence
said `Something went bad' and offered the choice to either `ignore', `retry' or `abort', none of which
buttons did react to mouse clicking).} As if that wasn't bad enough, those kind people that hang out
on discussion websites will likely start talking about how \TeX\ chews, digests, ruminates, and, well,
sometimes regurgitates its inputs when you go and ask them for help.

\TeX's behavior must be classified as singularly unhelpful and
irritating; you might want to say that \TeX\ cannot be successfully run without the Internet, for everything
save the shortest and most boring documents will, with a chance of near-certainty, have
{\em some} part---maybe something like a \verb#\usepackage{x}# statement that was inadvertently placed {\em
before} another, superficially unrelated \verb#\usepackage{y}# instead of {\em after} it---that makes \TeX\
screw up.

Well-meaning fellow \TeX-users will tell you to read {\em The \TeX\ Book}, but in case you can count you will
quickly see that---despite its purported reputation of being a fine manual---it actually is more a challenge to
read than anything else, and i'm specifically referring here to the fact that it has, on the roughly 300
pages of the main part, no less than 284 exercises (of which only 71 are `plain' exercises; 91 are labelled
`dangerous` and 122 `ddangerous'). To be fair, answers are given. Alas, when you peruse {\em Appendix I:
Index}, it'll make you think when you realize that there plainly is no keyword of interest with less than, say,
three or four, often six or ten page numbers next to it, meaning that to look up anything at all
you'll always have to sort of embark on a small grand tour. Add to that the author's penchant for a witty style
that often gets lost in obscure details\footnote{there are almost 30 pages dedicated to `Dirty Tricks';
that chapter is guarded with an eleven-fold `Danger!'-sign.} before getting off at a tangent and you'll more often than not rather inquire
about a problem in the Net right away rather than consult {\em The Book}.

Maybe the best thing that can be said about \TeX\ is that `it is good at typesetting', which, ultimately, is
of course why so many people are ready to put up with all that \TeX\ is not so good at. Which is a lot, to
say the least. Like, adding numbers. Or, doing string processing. Or, doing the fundamental things that
distinguish a markup language from a programming language: naming, looping, branching, function building,
and sensible data types (minimalists will claim much less is needed for Turing-completeness, but practioners
know that with much fewer tools, doing things will become an inordinately arduous task).\footnote{It can at
times be bwildering to see people `doing it the hard way' when easier ways are readily available, and
why?---`just because they can'. While mental training is certainly a laudable way to spend time, i cannot
help but feeling that \TeX\ has a way to waste mine, for the precise reason that for all the effort i put
into mastering it i get back so little in terms of re-usable knowledge. In a similar way, many in the field
of CS are ready to jump from `it's Turing-complete' to `you can and should use it to write programs with'.
Hell, even CSS$+$HTML and certain card games may be considered Turing-complete (see
http://beza1e1.tuxen.de/articles/accidentally\_turing\_complete.html, http://boingboing.net/2012/09/12/magic-the-gathering.html).}

When i say functions, i mean functions, that is,

\begin{quote}
optionally named bundles of functionality which accept
certain classes of inputs (as arguments) and yield outputs (as return values) with or without having
so-called `side effects'.\footnote{see https://en.wikipedia.org/wiki/Side\_effect\_(computer\_science);
essentially: ``a function is said to have a side effect if ∆ it also modifies some state or has an
observable interaction with ∆ the outside world. For example, a function might ∆ modify one of its
arguments, raise an exception, write data to a display or file ∆.'' In practical computing, side effects are
important because we want to `do something' (change state) and `produce something' (namely output).}
\end{quote}

Noticeably absent from this definition is the ability to do the one thing when called in one environment,
and the other when called from within another environment---which is what too many \TeX\ macros actually do.
In a Real programming language, you do not want a function \verb#f = ( x ) -> ...# to return 42 when called
as \verb#5 / ( f 3 )# but 71 when called as \verb#( f 3 ) / 5# just because \verb#f# happened to have
been situated once in the divisor and once in the dividend position of an expression; that would be more
than weird. There sure {\em may} be cases where such a behavior makes sense, but those are few and far
between.\footnote{One exception to the rule is the output of many command line tools that will use colors
when writing to the monitor, but omit colors when writing to a file, as most viewers will fail to render
colors and instead display unsightly control codes.} In particular, given the  knowledge that \verb#f 3#
returns some number, you do not want that same call \verb#f 3# to blow up the VM with an error message
when used in the context of, say, a multiplication instead of a division.\footnote{except when an overflow
occurs}

\TeX\ macros frequently do exactly that, and while \LaTeX\ sure is a feat of a great thinker and many
ingenious and diligent minds have struggled to come up with useful packages to make typesetting with \TeX\
easier, presumably few have succeeded in providing `real' functions in the sense outlined above; in my
experience, there's always a good chance that a given command will break whenever used outside of the select
environments it has been tested in / was intended for.

As a case in point, consider the monospaced words appearing in some of the section
titles of the document you're now reading, like `\thinspace{\em The \texttt{aux} Object}'. In most cases where the
monospaced font appears, the \verb!\verb#foo#! command has been used (in fact, the last instance of monospaced
text appears as \verb°\verb!\verb#foo#!° in the source; now you know what style {\em The \TeX\ Book} is written
in). But \verb!\verb#x#! happens to break when used in the argument to a \verb#\subsection{}#
command:

\begin{verbatim}
\subsection{The \verb#aux# Object}
\end{verbatim}

The above just leads to tears:

\begin{verbatim}
Illegal parameter number in definition of \GetTitleStringResult.
\end{verbatim}

Incidentally, you can't use \verb#\verb# in a \verb#\footnote{}#, either, but this time you get
a completely unrelated and equally opaque message:

\begin{verbatim}
You can't use `macro parameter character #' in horizontal mode.
\end{verbatim}

I have no idea why there are quotes in this phrase, but maybe we've hit upon something here: Both messages
share the word `parameter', and indeed, \verb°#° is \TeX's way to mark parameters.\footnote{in its own
in-imitable ways: \texttt{\#1} refers to the first argument to a macro, but \texttt{\#\#1} must be used
inside another macro that is nested inside that first macro to refer back to the same argument. In case
you should care: you'd use \texttt{\#\#\#\#1} to refer to the same argument if you were to try
and define a macro nested inside a nested macro, {\em not} \texttt{\#\#\#1}.}

OK great, not sure why that snazzy \verb°#° works in a plain paragraph but not in arguments to (some? all?)
commands, but, hey, we sure can use another character in its place---after all, \verb§\verb#x#§ can be
rewritten as \verb§\verb&x&§ or \verb§\verbXxX§ (or with any other hedge character, so we can always find
one that does not interfere with the text meant to be displayed), so, sure enough!, we can drop that
problematic \verb°#° and swap it for something else, can we?

\begin{verbatim}
\footnote{The \verb+aux+ Object}
\end{verbatim}

Nope. Gets you this:

\begin{verbatim}
Missing } inserted.
\end{verbatim}

Ah well. Let's try that with \verb#\subsection# then:

\begin{verbatim}
\subsection{The \verb+aux+ Object}
\end{verbatim}

Nope. Gets you this:

\begin{verbatim}
LaTeX Error: \verb illegal in command argument.
\end{verbatim}

I tried this with several hedge characters but to no avail. Needless to say the kind people that hang out on
websites dedicated to solving problems with \TeX\ do provide solutions. Also needless to say that the
respective solutions suggested for \verb#\footnote# and \verb#\subsection# that i saw are rather orthogonal
to each other---apparently, the observed breakages are due to different aspects of the underlying code that
define \verb#\footnote# and \verb#\subsection#.

Let's face it, programming is hard. Programming gets even harder when you have to work against the grain
with every step you take. Programming in \TeX\ is especially hard because of the insanely high degree of
complectedness\thinspace\footnote{Watch Rich Hickey on http://www.infoq.com/presentations/Simple-Made-Easy}
the language features. What you do to manage the considerable overhead that comes with a system of this
complexity is using so-called `design patterns', that is, bluntly, `coping strategies for programmers',
and, even more bluntly, `cop{\em y}ing strategies'.

Programming \TeX, it would seem, is
really only manageable (to a degree) by doing {\em nothing but} applying design patterns, precisely because
the language is only marginally more ergonomic than, say, INTERCAL.\footnote{see
https://en.wikipedia.org/wiki/INTERCAL\_programming\_language}

In fact, it has been said that design patterns---while certainly being
helpful in that they show people how to tackle classes of problems---are really `anti-patterns': signs that
a given language is poorly suited for a given way of doing things.

Art Atwood puts it this way:

\begin{quote}
If you find yourself frequently writing a bunch of boilerplate design pattern code to deal with
a ``recurring design problem'', that's not good engineering-- it's a sign that your language is fundamentally
broken. ∆ [design patterns] turn the programmer into a fancy macro processor.\footnote{see
http://www.codinghorror.com/blog/2007/07/rethinking-design-patterns.html}
\end{quote}

One comment i read stated that its writer could only come to grips with \TeX\ by way of `uttering
incantations' (i.e. copying black-box solutions found on the Internet and keeping fingers crossed they'll
just work); in other words, people are---by the very design of the environment they're put into---encouraged
to resort to that malicious, bad habit that is Cargo Cult Programming (which
in this world must be having more adherents than its close cousin, Object Oriented Programming).

Donald Knuth, famous inventor of \TeX, writes, on page 400 in the aforementioned chapter `Dirty Tricks',
that ``[i]t would be possible to write an entire book about \TeX\ ouput routines''; this is probably more
than a mild understatement in the face of comments like these:\footnote{recommended reading:
http://tex.stackexchange.com/a/8736/28067; this answer contains links to a four-part article series in the
magazine {\em TUGboat} which may or may not substantiate the implicit claim made here (that some parts of
\TeX are  too hard even for the most dedicated).}

\begin{verbatim}
% We empty any left over kludge insert box here; this is a temporary fix.
% It should perhaps be applied to one page of cleared floats, but
% who cares?  The whole of this stuff needs completely redoing for
% many such reasons.
\end{verbatim}

which anyone with a \LaTeX installation can read on their own machine by opening \verb#ltoutput.dtx# in
their text editor. These words (alongside with other delicious commentaries left by members of the \LaTeX\
dev team) are both testament to the inner complexity of \TeX\ and the limited capabilities of us frail
humans, beings who can only juggle with so many balls and hold on to only so many loose ends at any given time.



Let us pinpoint a few observations on what makes \TeX\ so hard.

XXXX XXXX XXXX XXXX XXXX XXXX XXXX XXXX XXXX XXXX XXXX XXXX XXXX XXXX XXXX XXXX XXXX XXXX XXXX

no namespacing

catcodes (active characters would have sufficed)

macro expansion,
\verb#\expandafter#, \verb#\noexpand#

dependency of outcome / success of macros on environment they are called from

no simple formulation of looping and branching

% 26215 occurrences of `\if.*` constructs = 245881 bytes / characters
% 1800 occurrences of single `\if` vs. 24415 occurrences of compound `\if.+` commands
% 5815 different `\if.*` commands

% \if@after
% \if@aftercode
% \if@afterindent
% \if@afternarr
% \if@aftersctnhead
% \if@aiaa@handcarry
% \if@aiaa@submit
% \if@allcaps
% \if@allincludes
% \if@AllowHfloat
% \if@allowindent
% \if@AllowSubFigure
% \if@altsecnumformat
% \if@alwaysbarcodes
% \if@ams
% \if@anim@@autopause
% \if@anim@@autoresume
% \if@anim@@dvipdfmx
% \if@anim@@dvips
% \if@anim@@loop
% \if@anim@autopause
% \if@anim@autoresume
% \if@anim@chfps
% \if@anim@controls
% \if@anim@draft
% \if@anim@grxloaded
% \if@anim@loop
% \if@anim@meas
% \ifMTOOLS
% \ifMTr@triangle
% \ifmu@H
% \ifmulticitation
% \ifmulticols
% \ifmultiline@
% \ifmultipleTBLheads
% \ifmultivolume
% \ifmultmore
% \ifmy@frame
% \ifmynotesinglepar
% \ifmynoteson
% \ifmynotestwocol
% \ifmytemp
% \ifmytest
% \ifzero
% \ifZeroExist
% \ifzeropt
% \ifzhfont@fakebold
% \ifzhs@isvalidchar
% \ifzhs@notactivehanzi
% \ifZREF@found
% \ifZREF@pa@list
% \ifzref@titleref@expand



\endgroup
